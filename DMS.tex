%Präambel:
\documentclass[parskip=half]{report}
\usepackage[ngerman]{babel}
\usepackage[utf8]{inputenc}				% für Sonderzeichen
\usepackage[T1]{fontenc}				% zur Umsetzung von Sonderzeichen
\usepackage{url}
\usepackage[pdfborder ={0 0 0}]{hyperref}		% für Verlinkungen



\usepackage{amsmath}					% Math. Zeichen
\usepackage{amssymb}					% Math. Symbole

\usepackage{graphicx}					% Bilder einfügen

\usepackage{longtable}					% Tabellen über eine Seite


\begin{document}

\chapter{Durchführung und Auswertung}
	Für die durchzuführenden Messungen benötigten wir nachfolgende Messgeräte:
		\begin{itemize}
		\item Peak Tech 2155 LCR Meter
		\item MX22 Multimeter
		\item KEITHLEY 2100 Digital Multimeter
		\item Hameg HM 8040 (Netzgerät für Gleichspannung U$_{S}$)
		\end{itemize}
	Zudem waren am Versuchsplatz ein Steckbrett, kleine Steckverbindungen, Kabel mit und ohne vergoldete Kontakte, die Dehnungsmessstreifen (4 Stück, angebracht auf einem "`Balken"'), diverse (Dünnschichtmetall-, Dickschichtmetall-, Graphit-) Widerstände und ein Kältespray. 
	
	Vor dem eigentlichen Versuchsbeginn prüften wir, inwiefern Kabel und Messgeräte Einfluss auf die Genauigkeit unserer Messung nehmen konnten. Mithilfe des "`Peak Tech 2155 LCR Meters"', welches Widerstände mit einer Genauigkeit von $0.1\%\pm1 $Digit im Bereich 1 - 0.1$\Omega$ misst, bestimmten wir durch 4-Draht-Messung die Widerstände unserer Kabel mit vergoldeten und normalen Kontakten. Dabei stellte sich heraus, dass diese im Bereich von (0.003$\pm0.001)\Omega$ bis (0.005$\pm0.001)\Omega$ lagen und somit als vernachlässigbar klein angenommen werden konnten. Bei der Bestimmung der Messabweichung für eine am Hameg HM 8040 eingestellten Spannung von U$_{S}$=10 V mithilfe des MX 22, zeigte unser Gerät eine Spannung von (10,06$\pm0.03)$V - die des Nachbartisches eine von (10.03$\pm0.03)$V an. Die Abweichung beträgt also ganze 0.03V und liegt damit in dem Bereich des Gerätefehlers des MX22 von $\pm0.3\%\pm3D$.
	
Bei der einfachen Vermessung eines blauen Dünnschicht-Metall-Widerstandes testeten wir die Abweichungen zwischen den verschiedenen Multimetern. Da man der Bedienungsanweisung der Geräte entnehmen kann, dass das KEITHLEY 2100 den geringsten Gerätefehler aufweist, ist der davon angezeigten Widerstand mit R=(119.9$\pm 0.023)\Omega$ in 2-Draht-Messung und R=(119.88$\pm 0.023 )\Omega$ in 4-Draht-Messung auch der Genaueste. Das Peak Tech 2155 bestimmte denselben Widerstand als R=(199.8$\pm0.130)\Omega$ und das MX22 als R=(119.7$\pm0.639)\Omega$. Es empfiehlt sich also für die genaueren Messungen, wie z.B. die Änderung der Brückenspannung für unterschiedliche Belastungen der Waage (A 1.4-1.6)das KEITHLEY 2100 zu verwenden, während bei den Aufgaben mit eher qualitativem Charakter auch das MX 22, bzw. das Peaktech 2155 aussagekräftige Messergebnisse liefern. Zugleich zeigte sich aber auch, dass die Verbesserung des Messergebnisses des KEITHLEYs von 2-Draht- auf 4-Draht-Messung mit $\Delta R=0.02\Omega$ neben den ohnehin bestehenden Messungenauigkeiten aus Gerätefehlern und Kabelfehlern unwesentlich wird. 

In Aufgabe 1.1 war nun eine Widerstandsmessung für einen Brückenwiderstand und ein DMS hinsichtlich ihrer Temperaturempfindlichkeit durchzuführen. Dafür nahmen wir mit dem PeakTech 2155 und dem MX22 Multimeter für drei Temperaturen, und zwar für Raumtemperatur $T\approx297.2K$, Körpertemperatur T$\approx309.2K$ und $T\approx223.2K$ (mit dem Kältespray), die Widerstände auf.


	
	

\end{document}