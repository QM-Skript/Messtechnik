%Präambel:
\documentclass{article}						% Vor dem Drucken auf scrartcl ändern
\usepackage[ngerman]{babel}
\usepackage[utf8]{inputenc}					% für Sonderzeichen
\usepackage[T1]{fontenc}					% zur Umsetzung von Sondereichen
\usepackage{url}
\usepackage[pdfborder ={0 0 0}]{hyperref}	% für Verlinkungen
\usepackage{enumitem}						% Zum Nummerieren
\usepackage{amsmath}						% Math. Zeichen
\usepackage{amssymb}						% Math. Symbole
\usepackage{graphicx}						% Bilder einfügen
\usepackage{longtable}						% Tabellen über eine Seite
\usepackage{subfigure}						% Zwei Bilder nebeneinander
\usepackage{wrapfig}
\usepackage{caption}						% Hilfsmittel, damit Bilder nicht auf die nächste Seite geschoben werden
%\usepackage[decimalsymbol=comma]{siunitx}				% Für schöne Einheiten mit Komma
            


\title{Lock-In-Verstärker}
\author{Friedrich Möller und Wilhelm Eschen}

\begin{document}
	\maketitle
	\tableofcontents
	\clearpage
			

\section{Aufgaben}

	
\clearpage

\section{Grundlagen}
		
		
	\clearpage
				 
\section{Auswertung}		
		
		\begin{figure}[h!]
			\centering
			\includegraphics[scale=0.3]{Linearitaet}
			\caption{Überprüfung der Linearität des Systems}
		\end{figure}
	\newline
		
	\clearpage

\section{Diskussion}
				
\end{document}